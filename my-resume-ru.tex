\documentclass[a4paper, 12pt]{article}
\usepackage[cm]{fullpage}
\usepackage[T2A]{fontenc}
\usepackage[utf8]{inputenc}

\usepackage{hyperref}
\hypersetup{
    pdftitle={Резюме},
    pdfauthor={Русакович Михаил Андреевич},
    pdfcreator={Русакович Михаил Андреевич},
    colorlinks=true,
    urlcolor=blue
}

\newcommand{\position}[1]{
    % bold
    \textbf{#1}}
\newcommand{\itemlabel}[1]{
    % italic + ":"
    \textit{#1:}}
\newcommand{\lastmodified}{
    \tiny{\itemlabel{Last Modified} \today}}

% Write current page number and Last Modified date in the footer
\usepackage{fancyhdr}
\pagestyle{fancyplain}
\fancyhf{}
\renewcommand{\headrulewidth}{0pt} % remove header line
\cfoot{\thepage}
\rfoot{\lastmodified}

\title{Русакович Михаил Андреевич}
\author{}
\date{}

\begin{document}
\maketitle

Обладаю высокой самомотивацией, высокой продуктивностью как в работе в команде, так и при самостоятельном выполнении заданий, и имею 3-х летний опыт в разработке enterprise- и web- ориентированного программного обеспечения в качестве Java-разработчика.\newline
Всегда очень заинтересован в освоении новых технологий, алгоритмов и просто наслаждаюсь видом произведений программного искусства. 

\subsubsection*{Контакты:}
\begin{itemize}
    \item Тел.: \href{tel:+375291798417}{+375 29 1798417}
    \item Email: \href{mailto:mikhail.complete@gmail.com}{mikhail.complete@gmail.com}
    \item Skype: \href{callto:mikhail.complete}{mikhail.complete}
    \item Полное имя: Русакович Михаил Андреевич
\end{itemize}


\section*{Навыки}

    \begin{itemize}
        \item Java, Groovy, C, PL/SQL (Oracle-диалект, включая оптимизацию запросов, иерархические запросы), JavaEE (JSF, EJB, JPA, Servlets, JSP, JAX-WS, JAX-RS, JNDI, JMS, CDI), JAXB, JCE, Spring, GWT, Hibernate, ActiveMQ, Apache CXF, JavaFX, Swing, Applets, JUnit, Mockito, Arquillian
        \item ООП, GOF-паттерны, Multithreading
        \item \itemlabel{Web-основы} HTTP, TCP/IP, HTML/CSS, Ajax, JavaScript, jQuery
        \item \itemlabel{Web- и контейнеры приложений} Weblogic, JBoss (WildFly), Tomcat, Glassfish
        \item \itemlabel{Спецификации} XML, XSD, XSLT, SOAP
        \item \itemlabel{Средства} Jenkins CI, Sonatype Nexus, Jira, SoapUI, Wireshark, SVN, Git, Maven, Ant, NSIS installation system
        \item \itemlabel{Твёрдое знание} ЭЦП, архитектуры CSP и security-провайдеров, PKCS7 и CMS форматов, работы с сертификатами и списками отзыва
    \end{itemize}


\section*{Опыт работы}

    \begin{itemize}
        \item \position{Java-разработчик}, Сентябрь 2012--Июль 2013, Декабрь 2014--настоящее время

            Разработка и поддержка TOPBY.by B2B провайдера EDI-сообщений, включающего в себя множество модулей конвертации и доставки сообщений  между бесчисленным числом клиентов по всей стране.

            Принимал активное участи в разработке архитектуры.
            	
            \begin{itemize}
                \item \itemlabel{Организация} ЗАО SoftClub, Минск, Беларусь
                \item \itemlabel{Обязанности} Разработка, тестирование, интеграция со внешними сервисами, 
                \newline профилирование, отладка, составление отчётов, ведение документации
                \item \itemlabel{Средства \& используемые технологии} Java, C, JavaScript, JNI, Weblogic, Tomcat, Arquillian, ActiveMQ, Apache CXF, Spring, Groovy, JPA, JDBC, JAXB, XML, XSD, XSLT, Applets, JCE, Swing, JavaFX, Maven, SVN, Jira, Nexus, NSIS Scriptable Install System
            \end{itemize}
            
            \textbf{Выполненная работа:}
			\begin{itemize}
  				\item развёрнут домен серверов приложений для высоконагруженной обработки сообщений;
  				\item разработано масштабируемое приложение обработки сообщений, используя существующий legacy-бизнес код;
				\item реализована ЭЦП-инфраструктура с поддержкой нескольких security-провайдеров на стороне клиента и сервера;
  				\item интеграция с партнерами используя SOAP, REST, JMS;
  				\item имплементация множества адаптирующих модулей из внутреннего формата клиента в EANCOM и обратно;
  				\item разработана система автоматической генерации договоров и отчетов по клиентам используя существующие шаблоны;
  				\item встроен и рабработан SSF-security модуль for SAP систем;
  				\item введена в эксплуатацию OAuth-авторизация для мобильных устройств;
			\end{itemize}
            
        \item \position{Java-разработчик}, Январь 2014--Декабрь 2014

          Хоум Кредит Банк (Беларусь) ibank-retail. Разработка интернет- банкинга для физических лиц. 

            \begin{itemize}
                \item \itemlabel{Организация} ЗАО SoftClub, Минск, Беларусь
                \item \itemlabel{Обязанности} Front-end и back-end разработка, отладка, bug fixing, тестирование, составление отчётов
                \item \itemlabel{Средства \& используемые технологии} Java, JSF (ICEfaces), JSP, EJB, IFX Framework, Hibernate, Oracle, JasperReports, JBoss, Dozer, Ant, jQuery, prototype.js, SVN, Jira
            \end{itemize}
            
               \textbf{Выполненная работа:}
			\begin{itemize}
  				\item разработан интерфейс на основе прототипа;
  				\item описан объектный маппинг в соответствии с моделью базы данных;
  				\item реализована бизнес-логика в соответствии со спецификацией;
				\item введена система составления отчётов по проведенным операциям;
  				\item интеграция со внешними сервисами с использованием SOAP, JMS;
			\end{itemize}


        \item \position{Java-разработчик}, Июль 2013--Январь 2014

            BUTB.by «Белорусская универсальная товарная биржа». Организация и проведение электронных биржевых торгов.

            \begin{itemize}
                \item \itemlabel{Организация} ЗАО SoftClub, Минск, Беларусь
                \item \itemlabel{Обязанности} Back-end и разработка сервисов, тестирование, составление отчётов, интеграция со внешними сервисами
                \item \itemlabel{Средства \& используемые технологии} Java, EJB, JPA, Servlets, Glassfish, Oracle PL/SQL, SOAP, Velocity, XDocReport, Groovy, Ant, SVN, Jira
            \end{itemize}
            
               \textbf{Выполненная работа:}
			\begin{itemize}
				\item экспозиция сервисов биржевых торгов через SOAP;
  				\item имплементация бизнес-логики в соответствии со спецификацией;
  				\item написание задач планирования для back-end части; 
				\item разработка скриптов для вспомогательных пакетных задач;
				\item интеграция с legacy-системами заказчика;
			\end{itemize}
    \end{itemize}    
    
\section*{Собственные открытые проекты}  

  
    \begin{itemize}
   
        \item \href{https://bitbucket.org/mikhail_complete/smarthome/src}{https://bitbucket.org/mikhail\_complete/smarthome}

            Проект SmartHouse - это попытка создать универсальную платформу управления различными устройствами используя популярные интерфейсы взаимодействия. В настоящий момент возможна работа с устройствами на основе последовательного, USB и 1-Wire протоколов и имеется большое число поддерживаемых устройств камер наблюдения разных производителей. Добавить поддержку своего собственного устройска крайне просто - нужно просто добавить его описание в один из конфигурационных файлов. Также, в арсенале есть большое число эмулирующих устройств для стадии разработки.\newline
            Для сложной логики управления и взаимодействия устройств был разработан специальный язык выражений. 
                        
            \begin{itemize}
                \item \itemlabel{Средства \& используемые технологии} Java, C, GWT,  Spring, RMI, JNI, Bootstrap, Jetty, JQuery, WebSocket, webcam-capture, Maven, JUnit, Mockito, Cucumber, Selenium WebDriver, Git
            \end{itemize}
            

        \item \href{https://github.com/creepid/DocsReporter}{https://github.com/creepid/DocsReporter}

            Система составления отчётов в форматах docx, odt, pdf, xhtml и pptx расширяющая XDocReport проект и использующая Spring Framework. Позволяет использовать уже существующие шаблоны вместо создания своего собственного. \newline Поддерживаемые форматы шаблонов: .docx, .odt и .pptx

            \begin{itemize}
                \item \itemlabel{Средства \& используемые технологии} Java, Spring, Velocity, Freemarker, JUnit, Mockito, Maven, Git
            \end{itemize}
            
              \item \href{https://github.com/creepid/jusbrelay}{https://github.com/creepid/jusbrelay}

            Мультиплатформанная Java библиотека для управления USB реле. \newline Поддерживаемые платформы: Windows, Linux, Apple OS X. \newline  Также доступна Python библиотека для тестовых целей.

            \begin{itemize}
                \item \itemlabel{Средства \& используемые технологии} Java, C, Python, JNI, JUnit, Maven, Git
            \end{itemize}
            
            \item \href{https://github.com/creepid/capicom-wrapper}{https://github.com/creepid/capicom-wrapper}

           Java-обёртка для популярной библиотеки работы с security-провайдерами Microsoft Сapicom.

            \begin{itemize}
                \item \itemlabel{Средства \& используемые технологии} Java, Jacob, COM, CSP, JUnit, Maven, Git
            \end{itemize}
            
    \end{itemize}    

\section*{Образование}

    \begin{itemize}

        \item \position{Белорусский Государственный Университет}, 2008--2013

            Научно-исследовательская деятельность.
             \newline\textbf{Тема дипломной работы:} 
             \newline «Разработка программного обеспечения управления ЭПР спектрометром»
    \end{itemize}
    
\section*{Иностранные языки}
 	\begin{itemize}
 		 \item \position{Английский язык}
 		 
 		 Разговорный, письменный уровня Upper-Intermediate. Чтение профессиональной технической документации.  
 	\end{itemize} 

\section*{Приложение}

    \begin{itemize}
        \item \href{https://github.com/creepid}{https://github.com/creepid}
    \end{itemize}
        \begin{itemize}
        \item \href{https://bitbucket.org/mikhail_complete}{https://bitbucket.org/mikhail\_complete}
    \end{itemize}

\end{document}
